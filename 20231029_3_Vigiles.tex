% !TEX TS-program = lualatex
% !TEX encoding = UTF-8

\documentclass[RG2023_CarnetSpecial.tex]{subfiles}

\ifcsname preamble@file\endcsname
  \setcounter{page}{\getpagerefnumber{M-20231029_3_Vigiles}}
\fi

\begin{document}

\bigtitle{30\superscript{e} dimanche \emph{per Annum} --- Vigiles}{~}

\smallscore{DLM_OF}
\translation{\vvrub Seigneur, ouvre mes lèvres.
\rrrub Et ma bouche annoncera ta louange.
Gloire au Père, au Fils, et au Saint-Esprit.
Comme il était au commencement, maintenant et toujours, 
et dans les siècles des siècles. Amen. Alléluia.}

\smalltitle{Invitatoire}

\gscore{F1I}
\translation{
Venez, crions de joie pour le Seigneur, acclamons notre Rocher, notre salut !
Allons jusqu'à lui en rendant grâce, par nos hymnes de fête acclamons-le !

Oui, le grand Dieu, c'est le Seigneur, le grand roi au-dessus de tous les dieux :
il tient en main les profondeurs de la terre, et les sommets des montagnes sont à lui ;

à lui la mer, c'est lui qui l'a faite, et les terres, car ses mains les ont pétries.
Entrez, inclinez-vous, prosternez-vous, adorons le Seigneur qui nous a faits.
Oui, il est notre Dieu ; nous sommes le peuple qu'il conduit, le troupeau guidé par sa main.

Aujourd'hui écouterez-vous sa parole ?
« Ne fermez pas votre coeur comme au désert, comme au jour de tentation et de défi,
où vos pères m'ont tenté et provoqué, et pourtant ils avaient vu mon exploit.

« Quarante ans leur génération m'a déçu, + et j'ai dit : Ce peuple a le coeur égaré, il n'a pas connu mes chemins.
Dans ma colère, j'en ai fait le serment : Jamais ils n'entreront dans mon repos. »}

\smalltitle{Hymne}
\gscore{F1OLH}
\hymntranslation{
Voici le temps de la minuit
où le prophète nous invite
à chanter à Dieu nos louanges,
au Père, au Fils, au Saint-Esprit.

Honorons-les pareillement :
à jamais nous devons louer
Dieu en sa Trinité parfaite
et dans son unique substance.

Voici le temps de l’épouvante
et de l’ange exterminateur
qui porta la mort en Egypte
et fit périr les premiers-nés.

C’est l’heure du salut des justes :
c’est en ce temps, au même lieu,
que l’ange n’osa les frapper,
respectant le signe du sang.

L’Égypte se lamente et pleure
l’affreuse mort de tant des siens ;
seul Israël connaît la joie,
gardé par le sang de l’agneau.

C’est nous l’Israël véritable :
repoussant l’assaut du Mauvais,
nous exultons en toi, Seigneur,
défendus par le sang du Christ.

Roi de sainteté, rends-nous dignes
de ton royaume et de sa gloire,
pour que nous puissions te chanter
en des louanges éternelles.}

\smalltitle{Psaume 103, i}
\gscore{H2F1OLA1}
\translation{Revêtu de magnificence, tu as pour manteau la lumière!}
\psalm{103-1}{TODO}

\smalltitle{Psaume 103, ii}
\gscore{H2F1OLA2}
\translation{Seigneur mon Dieu, tu es si grand!}
\psalm{103-1}{TODO}

\smalltitle{Psaume 103, iii}
\gscore{H2F1OLA3}
\translation{Quelle profusion dans tes oeuvres, Seigneur !}
\psalm{103-1}{TODO}

\smalltitle{Verset}

\twocoltext{\vv TODO. \\
\rr TODO. \\}{
\vv Heureux vos yeux puisqu’ils voient.\\
\rr Et vos oreilles puisqu’elles entendent.}

\smalltitle{Lectures et Répons}

Incipit liber Sapiéntiæ.

Aimez la justice, vous qui gouvernez la terre,
	ayez sur le Seigneur des pensées droites, cherchez-le avec un cœur simple,
	car il se laisse trouver par ceux qui ne le mettent pas à l’épreuve,
	il se manifeste à ceux qui ne refusent pas de croire en lui.
Les pensées tortueuses éloignent de Dieu,
	et sa puissance confond les insensés qui la provoquent.
Car la Sagesse ne peut entrer dans une âme qui veut le mal,
	ni habiter dans un corps asservi au péché.

\gscore{R1}
\translation{\rrrub Au commencement, avant que Dieu fît la terre, avant qu’il constituât les abîmes, avant qu’il fît jaillir les sources d’eau, * Avant qu’il donnât leur place aux montagnes, avant toutes les collines, le Seigneur m’a engendrée.
\vvrub Quand il préparait les deux j’étais là, disposant toutes choses avec lui. * Avant...}

L’Esprit saint, éducateur des hommes, fuit l’hypocrisie,
	il se détourne des projets sans intelligence,
	quand survient l’injustice, il la confond.
La Sagesse est un esprit ami des hommes,
	mais elle ne laissera pas le blasphémateur impuni pour ses paroles;
	car Dieu scrute ses reins, avec clairvoyance il observe son cœur,
	il écoute les propos de sa bouche.
L’esprit du Seigneur remplit l’univers:
	lui qui tient ensemble tous les êtres, il entend toutes les voix.
C’est pourquoi nul n’est à l’abri lorsqu’il tient des propos injustes:
	la Justice qui confond les coupables ne l’épargnera pas.

\gscore{R2}
\translation{\rrrub Du cercle du ciel j’ai fait le tour, et sur les fiots de la mer j’ai marché ; sur toutes les races et tous les peuples j’ai tenu la primauté. * J’ai foulé aux pieds, par ma puissance, le cou des superbes et des grands.
\vvrub Moi, j’habite au plus haut des cieux, et mon trône est sur une colonne de nuée. * J’ai foulé...}

Sur les intentions de l’impie, il y aura une enquête,
	le bruit de ses paroles parviendra jusqu’au Seigneur
		qui le confondra pour ses forfaits.
Une oreille attentive écoute tout;
	même le murmure des récriminations ne reste pas caché.
Gardez-vous donc d’une récrimination inutile,
	et plutôt que de dire du mal, retenez votre langue,
	car un propos tenu en cachette ne restera pas sans effet:
	la bouche qui calomnie détruit l’âme.
Ne courez pas après la mort en dévoyant votre vie, 
	n’attirez pas la catastrophe par les œuvres de vos mains.
Dieu n’a pas fait la mort, 
	il ne se réjouit pas de voir mourir les êtres vivants.
Il les a tous créés pour qu’ils subsistent; 
	ce qui naît dans le monde est porteur de vie: 
	on n’y trouve pas de poison qui fasse mourir.
La puissance de la Mort ne règne pas sur la terre,
	car la justice est immortelle.

\gscore{R3}
\translation{\rrrub Envoyez la sagesse. Seigneur, du siège de votre grandeur, pour qu’elle soit avec moi et travaille avec moi, * Pour que je sache en tout temps ce qui vous agrée.
\vvrub Donnez-moi, Seigneur, la sagesse assistante de vos jugements.
* Pour...}

Ex Epístula sancti Cleméntis papæ Primi ad Corínthios.

Regardons attentivement le Père et Créateur du monde entier, 
	attachons-nous aux bienfaits magnifiques et insurpassables qu'il nous donne dans la paix.
Contemplons-le par la pensée
	et considérons avec les yeux de l'âme la longue patience de ses desseins ;
	comprenons combien il agit sans aucune colère envers toute sa création.
Les cieux se déplacent sous sa direction et lui obéissent dans la paix.
Le jour et la nuit accomplissent le parcours qu'il leur a fixé, sans se gêner réciproquement.
Le soleil, la lune et les constellations gravitent selon son ordre, harmonieusement,
	sans jamais franchir les limites qu'il leur a données.
La terre féconde, docile à sa volonté,
	fait naître en abondance, selon les saisons qui conviennent, la nourriture
		destinée aux hommes, aux animaux et à tous les vivants qui l'habitent.
Elle ne conteste pas, elle ne change pas les règles qu'il a posées.

\gscore{R4}
\translation{\rrrub Donnez-moi, Seigneur, la sagesse assistante de vos jugements, et ne me rejetez pas d’entre vos familiers :
* Car je suis votre serviteur et le fils de votre servante.
\vvrub Envoyez-la du siège de votre grandeur, pour qu’elle soit avec moi et travaille avec moi. * Car...}

Les profondeurs des abîmes et les régions inexplorées sont régies par les mêmes lois.
Le gouffre illimité de la mer a été organisé en bassins par son habileté créatrice
	et ne franchit pas les limites où il l'a enfermé
	mais obéit aux ordres qui lui ont été prescrits.
Car il a dit : Tu viendras jusqu'ici, et tes flots se briseront sans sortir de toi.
Les océans que l'homme ne peut franchir et les mondes qui sont au-delà
	obéissent aux mêmes lois du Maître.

\gscore{R5}
\translation{\rrrub Le commencement de la sagesse est la crainte du Seigneur : * Bien avisés sont tous ceux qui la pratiquent; sa gloire subsiste à jamais,
\vvrub L’amour de cette sagesse est la gardienne des lois : car toute sagesse a la crainte du Seigneur. * Bien avisés...}

Les saisons du printemps, de l'été, de l'automne et de l'hiver se succèdent paisiblement.
Les réceptacles des vents accomplissent leur rôle au moment voulu, sans broncher.
Les sources intarissables, créées pour le plaisir et pour la santé des hommes,
	ne cessent de leur présenter leurs mamelles vivifiantes.
Les plus petits des animaux se réunissent dans la concorde et la paix.
Le grand Créateur et Maître de l'univers
	a ordonné que tout cela se fasse dans la paix et la concorde,
	car il répand ses bienfaits sur tous, mais il les prodigue surabondamment pour nous,
	qui avons voulu nous confier en ses miséricordes par notre Seigneur Jésus Christ.
À lui gloire et majesté pour les siècles des siècles. Amen.

\gscore{R6}
\translation{\rrrub Deux Séraphins se criaient l’un à l’autre :
* Saint, saint, saint est le Seigneur Dieu des armées : * Toute la terre est pleine de sa gloire.
\vvrub Ils sont trois qui rendent témoignage dans le ciel : le Père, le Verbe et l’Esprit-Saint ; et ces trois sont une seule chose.
* Saint, saint, saint est le Seigneur Dieu des armées :
\vvrub Gloire au Père, au Fils, * et au Saint-Esprit.
* Toute la terre est pleine de sa gloire.}

\smalltitle{Cantiques}

\gscore{H2F1OLAC}
\translation{Voici notre Dieu, en lui nous espérions, et il nous a sauvés, alléluia.}

\rubric{Is 33: 2-10}

Dómine, miserére nostri, *
te enim exspectávimus;
esto bráchium nostrum in mane *
et salus nostra in témpore tribulatiónis.
A voce fragóris fugérunt pópuli, *
ab exaltatióne tua dispérsæ sunt gentes.
4 Et congregabúntur spólia, sicut collígitur bruchus; *
sicut discúrrunt locústæ, ad ea discúrritur.
5 Sublímis est Dóminus, quóniam hábitat in excélso; *
implet Sion iudício et iustítia.
6 Et erit fírmitas in tempóribus tuis; †
divítiæ salútis sapiéntia et sciéntia: *
timor Dómini ipse est thesáurus eius.
7 Ecce præcónes clamábunt foris, *
ángeli pacis amáre flebunt.
8 Dissipátæ sunt viæ, *
cessávit tránsiens per sémitam;
írritum fecit pactum, *
reiécit testes,
non reputávit hómines.
9 Luget et elanguéscit terra, *
confúsus est Líbanus et obsórduit,
et factus est Saron sicut desértum, *
et exaruérunt Basan et Carmélus.
10 « Nunc consúrgam, dicit Dóminus, *
nunc exaltábor, nunc sublevábor ».

\rubric{Is 33: 13-16}

Audíte, qui longe estis, quæ fécerim, *
et cognóscite, vicíni, fortitúdinem meam.
14 Contérriti sunt in Sion peccatóres, *
possédit tremor ímpios.
Quis póterit habitáre de vobis cum igne devoránte? *
Quis habitábit ex vobis cum ardóribus sempitérnis?
Qui ámbulat in iustítiis *
et lóquitur æquitátes,
qui réicit lucra ex rapínis *
et éxcutit manus suas, ne múnera accípiat,
qui obtúrat aures suas, ne áudiat sánguinem, *
et claudit óculos suos, ne vídeat malum:
16 iste in excélsis habitábit, *
muniménta saxórum refúgium eius;
panis ei datus est, *
aquæ eius fidéles sunt.

\rubric{Si 36: 14-19}

Miserére plebi tuæ,
super quam invocátum est nomen tuum, *
et Israel, quem coæquásti primogénito tuo.
15 Miserére civitáti sanctificatiónis tuæ, *
Ierúsalem, loco requiéi tuæ.
16 Reple Sion maiestáte tua *
et glória tua templum tuum.
17 Da testimónium his,
qui ab inítio creatúræ tuæ sunt, *
et súscita prædicatiónes,
quas locúti sunt in nómine tuo.
18 Da mercédem sustinéntibus te, *
ut prophétæ tui fidéles inveniántur.
Et exáudi oratiónes servórum tuórum, †
19 secúndum beneplácitum super pópulo tuo, *
et dírige nos in viam iustítiæ,
et sciant omnes, qui hábitant terram, *
quia tu es Deus sæculórum.

Léctio sancti Evangélii secúndum Ioánnem.

Le premier jour de la semaine, Marie Madeleine se rend au tombeau de grand matin ;
	c’était encore les ténèbres.
Elle s’aperçoit que la pierre a été enlevée du tombeau.
Elle court donc trouver Simon-Pierre et l’autre disciple, celui que Jésus aimait,
	et elle leur dit : « On a enlevé le Seigneur de son tombeau,
	et nous ne savons pas où on l’a déposé. »
Pierre partit donc avec l’autre disciple pour se rendre au tombeau.
Ils couraient tous les deux ensemble,
	mais l’autre disciple courut plus vite que Pierre et arriva le premier au tombeau.
En se penchant, il s’aperçoit que les linges sont posés à plat ; cependant il n’entre pas.
	Simon-Pierre, qui le suivait, arrive à son tour.
Il entre dans le tombeau ; il aperçoit les linges, posés à plat,
	ainsi que le suaire qui avait entouré la tête de Jésus,
	non pas posé avec les linges, mais roulé à part à sa place.
C’est alors qu’entra l’autre disciple, lui qui était arrivé le premier au tombeau.
Il vit, et il crut.
Jusque-là, en effet, les disciples n’avaient pas compris 
	que, selon l’Écriture, il fallait que Jésus ressuscite d’entre les morts.
Ensuite, les disciples retournèrent chez eux.
Marie Madeleine se tenait près du tombeau, au-dehors, tout en pleurs.
Et en pleurant, elle se pencha vers le tombeau.
Elle aperçoit deux anges vêtus de blanc, assis l’un à la tête et l’autre aux pieds,
	à l’endroit où avait reposé le corps de Jésus.
Ils lui demandent : « Femme, pourquoi pleures-tu ? »
Elle leur répond : « On a enlevé mon Seigneur, et je ne sais pas où on l’a déposé. »
Ayant dit cela, elle se retourna ;
	elle aperçoit Jésus qui se tenait là, mais elle ne savait pas que c’était Jésus.
Jésus lui dit : « Femme, pourquoi pleures-tu ? Qui cherches-tu ? »
Le prenant pour le jardinier, elle lui répond :
	« Si c’est toi qui l’as emporté, dis-moi où tu l’as déposé, et moi, j’irai le prendre. »
Jésus lui dit alors : « Marie ! »
	S’étant retournée, elle lui dit en hébreu : « Rabbouni ! », c’est-à-dire : Maître.
Jésus reprend : « Ne me retiens pas, car je ne suis pas encore monté vers le Père.
Va trouver mes frères
	pour leur dire que je monte vers mon Père et votre Père, vers mon Dieu et votre Dieu. »
Marie Madeleine s’en va donc annoncer aux disciples : « J’ai vu le Seigneur ! »,
	et elle raconta ce qu’il lui avait dit.
	
\end{document}