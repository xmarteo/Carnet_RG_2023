% !TEX TS-program = lualatex
% !TEX encoding = UTF-8

\documentclass[RG2023_CarnetCommun.tex]{subfiles}

\ifcsname preamble@file\endcsname
  \setcounter{page}{\getpagerefnumber{M-20231028_5_Messe}}
\fi

\begin{document}

\bigtitle{Saints Simon et Jude, apôtres --- Messe --- forme ordinaire}

\smalltitle{Introït}
\gscore{1028In}
\translation{\rubric{Ps. 138: 17, 1, 2, 3} Tes amis ont été honorés, ô Dieu ; leur dignité de princes a été puissamment établie.\\
\vv \rubric{\emph{1. }} Tu me scrutes, Seigneur, et tu sais ! Tu sais quand je m'assois, quand je me lève.\\
\vv \rubric{\emph{2. }} De très loin, tu pénètres mes pensées: que je marche ou me repose, tu le vois.\\
\vv \rubric{\emph{3. }} Tous mes chemins te sont familiers, avant qu'un mot ne parvienne à mes lèvres.}

\smallscore{MOR02_InNominePatris}
\translation{Au nom du Père, du Fils et du Saint-Esprit. \rrrub Amen.}

\smallscore{MOR03_GratiaDomini}
\translation{La grâce de Jésus notre Seigneur, l'amour de Dieu le Père et la communion de l'Esprit Saint soient toujours avec vous.
\rrrub Et avec votre esprit.}

\smalltitle{Acte pénitentiel}
\phantomsection\label{actepenitentiel}

\twocoltext{Fratres, agnoscámus peccáta nostra, 
ut apti simus ad sacra mystéria celebránda.}{Préparons nous à célébrer le mystère de l’Eucharistie, 
en reconnaissant que nous avons péché.}

\twocoltext{Confíteor Deo omnipoténti et vobis, fratres,
quia peccávi nimis
cogitatióne, verbo, ópere et omissióne: 
mea culpa, mea culpa, mea máxima culpa. 
Ideo precor beátam Maríam semper Vírginem,
omnes Angelos et Sanctos,
et vos, fratres, oráre pro me
ad Dóminum Deum nostrum.}{Je confesse à Dieu tout-puissant, 
je reconnais devant vous, frères et soeurs, que j’ai péché
en pensée, en parole, par action et par omission ; 
oui j’ai vraiment péché.
c’est pourquoi je supplie
la bienheureuse Vierge Marie,
les anges et tous les saints,
et vous aussi, frères et soeurs, 
de prier pour moi le Seigneur notre Dieu.}

\twocoltext{Misereátur nostri omnípotens Deus
et, dimissís peccátis nostris,
perdúcat nos ad vitam ætérnam. \rrrub Amen.}{
Que Dieu Tout-Puissant nous fasse miséricorde ; qu'il nous pardonne nos péchés et nous conduise à la vie éternelle. \rrrub Amen.}

\gscore{KY4K}

\gscore{KY4G}
\begin{alltt}\normalfont
\emph{Gloire à Dieu, au plus haut des cieux, 
et paix sur la terre aux hommes qu’il aime.
Nous te louons, nous te bénissons, nous t’adorons,
Nous te glorifions, nous te rendons grâce, pour ton immense gloire,
Seigneur Dieu, Roi du ciel, Dieu le Père tout-puissant.
Seigneur, Fils unique, Jésus Christ,
Seigneur Dieu, Agneau de Dieu, le Fils du Père.
Toi qui enlèves le péché du monde, prends pitié de nous
Toi qui enlèves le péché du monde, reçois notre prière ;
Toi qui es assis à la droite du Père, prends pitié de nous.
Car toi seul es saint, toi seul es Seigneur,
Toi seul es le Très-Haut, Jésus Christ, avec le Saint-Esprit
Dans la gloire de Dieu le Père. Amen.}
\end{alltt}

\smalltitle{Collecte}

\twocoltext{
\vv Orémus.}{
\vv Prions le Seigneur.}
\twocoltext{Deus, qui nos per beátos Apóstolos
ad agnitiónem tui nóminis veníre tribuísti,
intercedéntibus sanctis Simóne et Iuda,
concéde propítius, ut semper augeátur Ecclésia
increméntis in te credéntium populórum. Per Dóminum nostrum Jesum Christum, Fílium tuum: qui tecum vivit et regnat in unitáte Spíritus Sancti, Deus, per ómnia sǽcula sæculórum.\\
\rr Amen.}{
TODO.\\
\rr Amen.}

\smalltitle{Première lecture}

\twocoltext{
Lectio Epístolæ beati Páuli \\ Apóstoli ad Ephésios.}{
Lecture de la lettre de Saint Paul, Apôtre, aux Éphésiens.}

\begin{alltt}\normalfont
Frères,
    vous n’êtes plus des étrangers ni des gens de passage,
vous êtes concitoyens des saints,
vous êtes membres de la famille de Dieu,
    car vous avez été intégrés dans la construction
qui a pour fondations les Apôtres et les prophètes ;
et la pierre angulaire, c’est le Christ Jésus lui-même.
    En lui, toute la construction s’élève harmonieusement
pour devenir un temple saint dans le Seigneur.
    En lui, vous êtes, vous aussi, les éléments d’une même construction
pour devenir une demeure de Dieu par l’Esprit Saint.
\end{alltt}

\smallscore{MOR05_VerbumDomini2}
\translation{Parole du Seigneur. \rrrub Nous rendons grâce à Dieu.}

\smalltitle{Graduel}
\gscore{1028Gr}
\translation{\rubric{Ps. 44: 17-18} \rrrub Sur toute la terre tu feras d'eux des princes : ils feront vivre ton nom pour les âges des âges.
\vvrub À la place de tes pères se lèveront tes fils : que les peuples te rendent grâce, toujours, à jamais !}

\smalltitle{Alléluia}
\gscore{1028Al}
\translation{\rubric{Ps. 138: 17}\\
Alléluia, alléluia. Tes amis ont été honorés, ô Dieu ; leur dignité de princes a été puissamment établie. Alléluia.}

\smalltitle{Évangile}

\smallscore{MOR06_LectioSanctiEv}
\translation{\vvrub Le Seigneur soit avec vous. \rrrub Et avec votre esprit.
\vvrub Évangile de Jésus-Christ selon saint Luc. \rrrub Gloire à toi, Seigneur.}

\begin{alltt}\normalfont
En ces jours-là,
Jésus s’en alla dans la montagne pour prier,
et il passa toute la nuit à prier Dieu.
    Le jour venu,
il appela ses disciples et en choisit douze
auxquels il donna le nom d’Apôtres :
    Simon, auquel il donna le nom de Pierre,
André son frère,
Jacques, Jean, Philippe, Barthélemy,
    Matthieu, Thomas,
Jacques fils d’Alphée,
Simon appelé le Zélote,
    Jude fils de Jacques,
et Judas Iscariote, qui devint un traître.
    Jésus descendit de la montagne avec eux
et s’arrêta sur un terrain plat.
Il y avait là un grand nombre de ses disciples
et une grande multitude de gens
venus de toute la Judée, de Jérusalem,
et du littoral de Tyr et de Sidon.
    Ils étaient venus l’entendre
et se faire guérir de leurs maladies ;
ceux qui étaient tourmentés par des esprits impurs
retrouvaient la santé.
    Et toute la foule cherchait à le toucher,
parce qu’une force sortait de lui
et les guérissait tous.
\end{alltt}

\smallscore{MOR07_VerbumDominiEv}
\translation{\vvrub Parole du Seigneur. \rrrub Louange à toi, ô Christ.}

\smalltitle{Offertoire}

\gscore{1028Of}
\translation{\rubric{Ps. 18: 5, 2} Sur toute la terre en paraît le message, et la nouvelle, aux limites du monde. \vvrub Les cieux proclament la gloire de Dieu, le firmament raconte l'ouvrage de ses mains.}

\smalltitle{Prière sur les offrandes}

\twocoltext{\vv Oráte, fratres:
ut meum ac vestrum sacrifícium
acceptábile fiat apud Deum Patrem omnipoténtem.\\
\rr Suscípiat Dóminus sacrifícium de mánibus tuis
ad laudem et glóriam nóminis sui,
ad utilitátem quoque nostram
totiúsque Ecclésiæ suæ sanctæ.}{
\vv Priez, frères et sœurs : que mon sacrifice, et le vôtre, soit agréable à Dieu le Père tout-puissant.\\
\rr Que le Seigneur reçoive de vos mains ce sacrifice à la louange et à la gloire de son nom, pour notre bien et celui de toute l’Église.}

\twocoltext{Glóriam, Dómine, sanctórum apostolórum Simónis et Iudæ
perpétuam venerántes,
quǽsumus, ut vota nostra suscípias
et ad sacra mystéria celebránda nos digne perdúcas.
Per Christum Dóminum nostrum.\\
\rr Amen.}{
TODO.\\
\rr Amen.}

\smalltitle{Préface}

\smallscore{MOR10_PrefaceOF}
\emph{\vvrub Le Seigneur soit avec vous. \rrrub Et avec votre esprit.\\
\vvrub Élevons notre cœur. \rrrub Nous le tournons vers le Seigneur.\\
\vvrub Rendons grâce au Seigneur notre Dieu. \rrrub Cela est juste et bon.}

\twocoltext{Vere dignum et iustum est, TODO}{
TODO}

\smalltitle{Sanctus}

\gscore{KY4S}
\emph{Saint, Saint, Saint, le Seigneur, Dieu de l'Univers. Le ciel et la terre sont remplis de ta gloire. Hosanna au plus haut des cieux. Béni soit celui qui vient au nom du Seigneur. Hosanna au plus haut des cieux.}

\smalltitle{Canon romain}
\phantomsection\label{canon}

\twocoltext{
Te ígitur, clementíssime Pater,
per Iesum Christum, Fílium tuum,
Dóminum nostrum,
súpplices rogámus ac pétimus,
uti accépta hábeas
signat semel super panem et calicem simul, dicens: 
et benedícas \cc hæc dona, hæc múnera,
hæc sancta sacrifícia illibáta,
in primis, quæ tibi offérimus
pro Ecclésia tua sancta cathólica:
quam pacificáre, custodíre, adunáre
et régere dignéris toto orbe terrárum:
una cum fámulo tuo Papa nostro \rubric{N.}
et Antístite nostro \rubric{N.}
et ómnibus orthodóxis atque cathólicæ
et apostólicæ fídei cultóribus.}{
Toi, Père très aimant, nous te prions et te supplions par Jésus Christ, ton Fils, notre Seigneur, d’accepter et de bénir \cc ces dons et ces offrandes, sacrifice pur et saint, que nous te présentons avant tout pour ta sainte Eglise catholique : accorde-lui la paix et protège-la, daigne la rassembler dans l'unité et la gouverner par toute la terre ; nous les présentons en union avec  ton serviteur le Pape \rubric{N.}, notre évêque \rubric{N.} et tous ceux qui gardent fidèlement la foi catholique reçue des Apôtres.}

\twocoltext{Meménto, Dómine,
famulórum famularúmque tuárum \rubric{N.} et \rubric{N.}
et ómnium circumstántium,
quorum tibi fides cógnita est et nota devótio,
pro quibus tibi offérimus:
vel qui tibi ófferunt hoc sacrifícium laudis,
pro se suísque ómnibus:
pro redemptióne animárum suárum,
pro spe salútis et incolumitátis suæ:
tibíque reddunt vota sua
ætérno Deo, vivo et vero.}{Souviens-toi, Seigneur, de tes serviteurs et de tes servantes (de \rubric{N.} et \rubric{N.}) et de tous ceux qui sont ici réunis, dont tu connais la foi et l'attachement. Nous t'offrons pour eux, ou ils t'offrent pour eux-mêmes et tous les leurs ce sacrifice de louange, pour leur propre rédemption, pour la paix et le salut qu'ils espèrent ; et ils te rendent cet hommage, à toi, Dieu éternel vivant et vrai.}

\twocoltext{Communicántes,
et memóriam venerántes,
in primis gloriósæ semper Vírginis Maríæ,
Genetrícis Dei et Dómini nostri Iesu Christi, 
sed et beáti Ioseph, eiúsdem Vírginis Sponsi,
et beatórum Apostolórum ac Mártyrum tuórum,
Petri et Pauli, Andréæ,
(Iacóbi, Ioánnis,
Thomæ, Iacóbi, Philíppi,
Bartholomǽi, Matthǽi,
Simónis et Thaddǽi: 
Lini, Cleti, Cleméntis, Xysti,
Cornélii, Cypriáni,
Lauréntii, Chrysógoni,
Ioánnis et Pauli,
Cosmæ et Damiáni)
et ómnium Sanctórum tuórum;
quorum méritis precibúsque concédas,
ut in ómnibus protectiónis tuæ muniámur auxílio.
(Per Christum Dóminum nostrum. Amen.)}{
Unis dans une même communion, vénérant d’abord la mémoire de la bienheureuse Marie toujours Vierge, Mère de notre Dieu et Seigneur, Jésus Christ ; et celle de saint Joseph, son époux, les saints Apôtres et Martyrs Pierre et Paul, André, [Jacques et Jean, Thomas, Jacques et Philippe, Barthélemy et Matthieu, Simon et Jude, Lin, Clet, Clément, Sixte, Corneille et Cyprien, Laurent, Chrysogone, Jean et Paul, Côme et Damien] et tous les saints.
Nous t'en supplions, accorde-nous, par leur prière et leurs mérites, d'être, toujours et partout, forts de ton secours et de ta protection. [Par le Christ notre Seigneur. Amen.]}

\twocoltext{Hanc ígitur oblatiónem servitútis nostræ,
sed et cunctæ famíliæ tuæ,
quǽsumus, Dómine, ut placátus accípias:
diésque nostros in tua pace dispónas,
atque ab ætérna damnatióne nos éripi
et in electórum tuórum iúbeas grege numerári.
(Per Christum Dóminum nostrum. Amen.)}{
Voici donc l’offrande que nous présentons devant toi, nous tes serviteurs, et ta famille entière : 
Seigneur, dans ta bienveillance, accepte-la.
Assure toi-même la paix de notre vie, arrache nous à la damnation éternelle et veuille nous admettre au nombre de tes élus.}

\twocoltext{Quam oblatiónem tu, Deus, in ómnibus, quǽsumus,
benedíctam, adscríptam, ratam,
rationábilem, acceptabilémque fácere dignéris:
ut nobis Corpus et Sanguis fiat dilectíssimi Fílii tui,
Dómini nostri Iesu Christi. 
Qui, prídie quam paterétur, 
accépit panem in sanctas ac venerábiles manus suas,
et elevátis óculis in cælum
ad te Deum Patrem suum omnipoténtem,
tibi grátias agens benedíxit,
fregit,
dedítque discípulis suis, dicens:}{Seigneur Dieu, nous t’en prions, daigne bénir et accueillir cette offrande, accepte-la pleinement, 
rends la parfaite et digne de toi : qu’elle devienne pour nous le Corps et le Sang de ton Fils bien-aimé,
Jésus, le Christ, notre Seigneur. 
La veille de sa passion, il prit le pain dans ses mains très saintes et, les yeux levés au ciel, vers toi, Dieu, son Père tout-puissant, en te rendant grâce il dit la bénédiction,, le rompit, et le donna à ses disciples, en disant :}

\twocoltext{Accípite et manducáte ex hoc omnes:
hoc est enim Corpus meum,
quod pro vobis tradétur.}{Prenez, et mangez-en tous ceci est mon corps livré pour vous.}

\twocoltext{Símili modo, postquam cenátum est, 
accípiens et hunc præclárum cálicem
in sanctas ac venerábiles manus suas,
item tibi grátias agens benedíxit,
dedítque discípulis suis, dicens:}{De même après le repas,
il prit cette coupe incomparable dans ses mains très saintes ; 
et te rendant grâce à nouveau, il dit la bénédiction, et donna la coupe à ses disciples, en disant :}

\twocoltext{Accípite et bíbite ex eo omnes:
hic est enim calix Sánguinis mei
novi et ætérni testaménti,
qui pro vobis et pro multis effundétur
in remissiónem peccatórum.
Hoc fácite in meam commemoratiónem.}{
Prenez, et buvez-en tous, car ceci est la coupe de mon sang, le sang de l'Alliance nouvelle et éternelle, qui sera versé pour vous et pour la multitude en rémission des péchés. Vous ferez cela, en mémoire de moi.
}

\smallscore{MOR13_MysteriumFidei}
\translation{\vvrub Il est grand, le mystère de la foi. \rrrub Nous annonçons ta mort, Seigneur Jésus, nous proclamons ta résurrection, nous attendons ta venue dans la gloire.}

\twocoltext{Unde et mémores, Dómine,
nos servi tui,
sed et plebs tua sancta,
eiúsdem Christi, Fílii tui, Dómini nostri,
tam beátæ passiónis,
necnon et ab ínferis resurrectiónis,
sed et in cælos gloriósæ ascensiónis:
offérimus præcláræ maiestáti tuæ
de tuis donis ac datis
hóstiam puram,
hóstiam sanctam,
hóstiam immaculátam,
Panem sanctum vitæ ætérnæ
et Cálicem salútis perpétuæ.}{
Voilà pourquoi nous aussi, tes serviteurs, et ton peuple saint avec nous, faisant mémoire de la passion bienheureuse de ton Fils, Jésus, le Christ, notre Seigneur, de sa résurrection du séjour des morts et de sa glorieuse ascension dans le ciel, nous te présentons, Dieu de gloire et de majesté, cette offrande prélevée sur les biens que tu nous donnes, le sacrifice pur et saint, le sacrifice parfait, pain de la vie éternelle et coupe du salut.}

\twocoltext{Supra quæ propítio ac seréno vultu
respícere dignéris:
et accépta habére,
sícuti accépta habére dignátus es
múnera púeri tui iusti Abel,
et sacrifícium Patriárchæ nostri Abrahæ,
et quod tibi óbtulit summus sacérdos tuus Melchísedech,
sanctum sacrifícium, immaculátam hóstiam.}{
Et comme il t'a plu d'accueillir les présents de ton serviteur Abel le Juste, le sacrifice d’Abraham, notre père dans la foi, et celui que t'offrit Melchisédech, ton grand prêtre, oblation sainte et immaculée, regarde ces offrandes avec amour et, dans ta bienveillance, accepte-les.}

\twocoltext{Súpplices te rogámus, omnípotens Deus:
iube hæc perférri per manus sancti Angeli tui
in sublíme altáre tuum,
in conspéctu divínæ maiestátis tuæ;
ut, quotquot ex hac altáris participatióne
sacrosánctum Fílii tui Corpus et Sánguinem
sumpsérimus,
omni benedictióne cælésti et grátia repleámur.
(Per Christum Dóminum nostrum. Amen.)}{
Nous t'en supplions, Dieu tout-puissant : qu'elles soit portées par les mains de ton ange en présence de ta gloire, sur ton autel céleste, afin qu'en recevant ici, par notre communion à l'autel, le corps et le sang très saints de ton Fils, 
nous soyons comblés de la grâce et de toute bénédiction du ciel. [Par le Christ, Notre Seigneur, Amen.]}

\twocoltext{Meménto étiam, Dómine,
famulórum famularúmque tuárum \rubric{N.} et \rubric{N.},
qui nos præcessérunt cum signo fídei,
et dórmiunt in somno pacis.
Ipsis, Dómine, et ómnibus in Christo quiescéntibus,
locum refrigérii, lucis et pacis,
ut indúlgeas, deprecámur.
(Per Christum Dóminum nostrum. Amen.)}{
Souviens-toi aussi, Seigneur,
de tes serviteurs et de tes servantes (de \rubric{N.} et \rubric{N.}) qui nous ont précédés, marqués du signe de la foi, et qui dorment dans la paix. Pour eux et pour tous ceux qui reposent dans le Christ, nous implorons ta bonté, Seigneur : qu'ils demeurent dans la joie, la lumière et la paix. [Par le Christ, Notre Seigneur, Amen]}

\twocoltext{Nobis quoque peccatóribus fámulis tuis,
de multitúdine miseratiónum tuárum sperántibus,
partem áliquam et societátem donáre dignéris
cum tuis sanctis Apóstolis et Martýribus:
cum Ioánne, Stéphano,
Matthía, Bárnaba,
(Ignátio, Alexándro,
Marcellíno, Petro,
Felicitáte, Perpétua,
Agatha, Lúcia,
Agnéte, Cæcília, Anastásia)
et ómnibus Sanctis tuis:
intra quorum nos consórtium,
non æstimátor mériti,
sed véniæ, quǽsumus, largítor admítte.
Per Christum Dóminum nostrum.}{
Et nous, pécheurs, tes serviteurs, qui mettons notre espérance en ta miséricorde inépuisable, admets-nous dans la communauté des saints Apôtres et Martyrs, avec Jean Baptiste, Étienne, Matthias et Barnabé, [Ignace, Alexandre, Marcellin et Pierre, Félicité et Perpétue, Agathe, Lucie, Agnès, Cécile, Anastasie,] et tous les saints. Nous t’en prions, accueille-nous dans leur compagnie,
sans nous juger sur le mérite
mais en accordant largement ton pardon,
par le Christ, notre Seigneur.}

\twocoltext{
Per quem hæc ómnia, Dómine, 
semper bona creas, sanctíficas, vivíficas, benedícis,
et præstas nobis.}{
Par lui, tu ne cesses de créer tous ces biens,
tu les sanctifies, leur donnes la vie, les bénis
et nous en fais le don.}

\twocoltext{Per ipsum, et cum ipso, et in ipso,
est tibi Deo Patri omnipoténti,
in unitáte Spíritus Sancti,
omnis honor et glória
per ómnia sǽcula sæculórum.\\
\rr Amen.}{
Par lui, avec lui et en lui, à toi, Dieu le Père tout-puissant, dans l'unité du Saint-Esprit, tout honneur et toute gloire, pour les siècles des siècles.\\
\rr Amen.}

\smalltitle{Notre Père}

\smallscore{MOR20_PaterNoster}
\translation{Comme nous l'avons appris du Sauveur, et selon son commandement, nous osons dire :
\rrrub Notre Père qui es au cieux, que ton nom soit sanctifié ; que ton règne vienne ; que ta volonté soit faite sur la terre comme au ciel. Donne-nous aujourd'hui notre pain de ce jour ; pardonne-nous nos offenses comme nous pardonnons aussi à ceux qui nous ont offensés ; et ne nous laisse pas entrer en tentation ; mais délivre-nous du mal.}

\twocoltext{
Líbera nos, quǽsumus, Dómine, ab ómnibus malis,
da propítius pacem in diébus nostris,
ut, ope misericórdiæ tuæ adiúti,
et a peccáto simus semper líberi
et ab omni perturbatióne secúri:
exspectántes beátam spem
et advéntum Salvatóris nostri Iesu Christi.}{
Délivre nous de tout mal, Seigneur,
et donne la paix à notre temps :
soutenus par ta miséricorde,
nous serons libérés de tout péché,
à l’abri de toute épreuve,
nous qui attendons que se réalise
cette bienheureuse espérance :  
l’avénement de Jésus Christ, notre Sauveur.}

\smallscore{MOR23_QuiaTuum}
\translation{\rrrub Car c'est à toi qu'appartiennent le règne, la puissance et la gloire, pour les siècles des siècles.}

\twocoltext{Dómine Iesu Christe, qui dixísti Apóstolis tuis:
Pacem relínquo vobis, pacem meam do vobis:
ne respícias peccáta nostra,
sed fidem Ecclésiæ tuæ;
eámque secúndum voluntátem tuam
pacificáre et coadunáre dignéris.
Qui vivis et regnas in sǽcula sæculórum.\\
\rr Amen.}{
Seigneur Jésus-Christ, tu as dit à tes apôtres : « Je vous laisse la paix, je vous donne la paix », ne regarde pas nos péchés, mais la foi de ton Église ; pour que ta volonté s'accomplisse, donne-lui toujours cette paix et conduis-la vers l'unité parfaite, toi qui règnes pour les siècles des siècles. 
\rr Amen.}

\smallscore{MOR24_PaxDomini}
\translation{\vvrub Que la paix du Seigneur soit toujours avec vous.\\
\rrrub Et avec votre esprit.}

\smalltitle{Agnus Dei}

\gscore{KY4A}

\smalltitle{Communion}

\twocoltext{
Ecce Agnus Dei, ecce qui tollit peccáta mundi.
Beáti qui ad cenam Agni vocáti sunt.}{
Voici l’agneau de Dieu,
voici celui qui enlève les péchés du monde.
Heureux les invités au repas des noces de l’Agneau !}

\twocoltext{Dómine, non sum dignus, ut intres sub téctum meum,
sed tantum dic verbo, et sanábitur ánima mea.}{
Seigneur, je ne suis pas digne de te recevoir, mais dis seulement une parole et je serai guéri.}

\gscore{1028Co}
\translation{\rubric{Mt. 19: 28; Ps. 125: 1-4} Vous qui m’avez suivi, vous siégerez sur douze trônes pour juger les douze tribus d’Israël.\\
\vv \rubric{\emph{1. }} Quand le Seigneur ramena les captifs à Sion, nous étions comme en rêve.\\
\vv \rubric{\emph{2a. }} Alors notre bouche était pleine de rires, nous poussions des cris de joie.\\
\vv \rubric{\emph{2b. }} Alors on disait parmi les nations : « Quelles merveilles fait pour eux le Seigneur ! »\\
\vv \rubric{\emph{3. }} Quelles merveilles le Seigneur fit pour nous : nous étions en grande fête.\\
\vv \rubric{\emph{4. }} Ramène, Seigneur, nos captifs, comme les torrents au désert.
}

\smalltitle{Postcommunion}

\twocoltext{
\vv Orémus.\\
Percéptis, Dómine, sacraméntis,
súpplices in Spíritu Sancto deprecámur,
ut, quæ pro apostolórum Simónis et Iudæ
veneránda gérimus passióne,
nos in tua dilectióne consérvent.
Per Christum Dóminum nostrum.\\
\rr Amen.}{
\vv Prions le Seigneur.\\
TODO.\\
\rr Amen.}

\smalltitle{Envoi}

\smallscore{MOR25_Benedicat}

\end{document}