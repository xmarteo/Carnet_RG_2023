\bigtitle{Aspersion}

\gscore{MOR01_Asperges}
\translation{TODO}

\bigtitle{30\superscript{e} dimanche \emph{per Annum} --- Messe --- forme ordinaire}

\smalltitle{Introït}
\gscore{TO30In}
\translation{\rubric{Ps. 104: 3, 4, 1, 2, 5} Joie pour les coeurs qui cherchent Dieu ! Cherchez le Seigneur et sa puissance, recherchez sans trêve sa face.\\
\vv \rubrique{\emph{1. }} Rendez grâce au Seigneur, proclamez son nom, annoncez parmi les peuples ses hauts faits.\\
\vv \rubrique{\emph{2. }} Chantez et jouez pour lui, redites sans fin ses merveilles.\\
\vv \rubrique{\emph{5. }} Souvenez-vous des merveilles qu'il a faites, de ses prodiges, des jugements qu'il prononça.\\}

\rubric{Acte pénitentiel: voir page TODO}

\gscore{KY11K}

\gscore{KY11G}

\smalltitle{Collecte}

\twocoltext{
\vv Orémus.}{
\vv Prions le Seigneur.}
\twocoltext{Omnípotens sempitérne Deus,
da nobis fídei, spei et caritátis augméntum,
et, ut mereámur ássequi quod promíttis,
fac nos amáre quod prǽcipis.
Per Dóminum nostrum Jesum Christum, Fílium tuum: qui tecum vivit et regnat in unitáte Spíritus Sancti, Deus, per ómnia sǽcula sæculórum.\\
\rr Amen.}{
TODO.\\
\rr Amen.}

\smalltitle{Première lecture}

\twocoltext{
Lectio Libri Exódi.}{
Lecture du livre de l’Exode.}

\begin{alltt}\normalfont
Ainsi parle le Seigneur :
    « Tu n’exploiteras pas l’immigré,
tu ne l’opprimeras pas,
car vous étiez vous-mêmes des immigrés au pays d’Égypte.
    Vous n’accablerez pas la veuve et l’orphelin.
    Si tu les accables et qu’ils crient vers moi,
j’écouterai leur cri.
    Ma colère s’enflammera et je vous ferai périr par l’épée :
vos femmes deviendront veuves, et vos fils, orphelins.

    Si tu prêtes de l’argent à quelqu’un de mon peuple,
à un pauvre parmi tes frères,
tu n’agiras pas envers lui comme un usurier :
tu ne lui imposeras pas d’intérêts.
    Si tu prends en gage le manteau de ton prochain,
tu le lui rendras avant le coucher du soleil.
    C’est tout ce qu’il a pour se couvrir ;
c’est le manteau dont il s’enveloppe,
la seule couverture qu’il ait pour dormir.
S’il crie vers moi, je l’écouterai,
car moi, je suis compatissant ! »
\end{alltt}

\smallscore{MOR04_VerbumDomini1}
\translation{Parole du Seigneur. \rrrub Nous rendons grâce à Dieu.}

\smalltitle{Graduel}
\gscore{TO30Gr}
\translation{\rubric{Ps. 26: 4} \rrrub J'ai demandé une chose au Seigneur, la seule que je cherche : habiter la maison du Seigneur tous les jours de ma vie. \vvrub Pour admirer le Seigneur dans sa beauté et m'attacher à son temple.}

\smalltitle{Deuxième lecture}

\twocoltext{
Lectio Epístolæ beati Páuli \\ Apóstoli ad Thessalonicénses.}{
Lecture de la lettre de Saint Paul, Apôtre, aux Thessaloniciens.}

\begin{alltt}\normalfont
Frères,
    vous savez comment nous nous sommes comportés chez vous
pour votre bien.
    Et vous-mêmes, en fait, vous nous avez imités, nous et le Seigneur,
en accueillant la Parole au milieu de bien des épreuves,
avec la joie de l’Esprit Saint.
    Ainsi vous êtes devenus un modèle pour tous les croyants
de Macédoine et de Grèce.
    Et ce n’est pas seulement en Macédoine et en Grèce
qu’à partir de chez vous la parole du Seigneur a retenti,
mais la nouvelle de votre foi en Dieu s’est si bien répandue partout
que nous n’avons pas besoin d’en parler.
    En effet, les gens racontent, à notre sujet,
l’accueil que nous avons reçu chez vous ;
ils disent comment vous vous êtes convertis à Dieu
en vous détournant des idoles,
afin de servir le Dieu vivant et véritable,
    et afin d’attendre des cieux son Fils
qu’il a ressuscité d’entre les morts,
Jésus, qui nous délivre de la colère qui vient.
\end{alltt}

\smallscore{MOR05_VerbumDomini2}
\translation{Parole du Seigneur. \rrrub Nous rendons grâce à Dieu.}

\smalltitle{Alléluia}
\gscore{TO30Al}
\translation{\rubric{Ps. 147: 1}\\
\emph{Alléluia, alléluia. Glorifie le Seigneur, Jérusalem : célèbre ton Dieu, ô Sion. Alléluia.}

\smalltitle{Évangile}

\smallscore{MOR06_LectioSanctiEv}
\translation{\vvrub Le Seigneur soit avec vous. \rrrub Et avec votre esprit.
\vvrub Évangile de Jésus-Christ selon saint Matthieu. \rrrub Gloire à toi, Seigneur.}

\begin{alltt}\normalfont
En ce temps-là,
    les pharisiens,
apprenant que Jésus avait fermé la bouche aux sadducéens,
se réunirent,
    et l’un d’entre eux, un docteur de la Loi, posa une question à Jésus
pour le mettre à l’épreuve :
    « Maître, dans la Loi,
quel est le grand commandement ? »
    Jésus lui répondit :
« Tu aimeras le Seigneur ton Dieu
de tout ton cœur,
de toute ton âme et de tout ton esprit.
    Voilà le grand, le premier commandement.
    Et le second lui est semblable :
Tu aimeras ton prochain comme toi-même.
    De ces deux commandements
dépend toute la Loi, ainsi que les Prophètes. »
\end{alltt}

\smallscore{MOR07_VerbumDominiEv}
\translation{\vvrub Parole du Seigneur. \rrrub Louange à toi, ô Christ.}

\smalltitle{Credo}

\gscore{KYCredo1}
\begin{alltt}\normalfont
\emph{Je crois en un seul Dieu, le Père tout puissant,
	créateur du ciel et de la terre, de l’univers visible et invisible,
Je crois en un seul Seigneur, Jésus Christ,
	le Fils unique de Dieu, né du Père avant tous les siècles :
Il est Dieu, né de Dieu, lumière, née de la lumière, vrai Dieu, né du vrai Dieu.
Engendré non pas créé, consubstantiel au Père ; et par lui tout a été fait.
Pour nous les hommes, et pour notre salut, il descendit du ciel;
Par l’Esprit Saint, il a pris chair de la Vierge Marie, et s’est fait homme.
Crucifié pour nous sous Ponce Pilate, 
	il souffrit sa passion et fut mis au tombeau.
Il ressuscita le troisième jour, conformément aux Ecritures,
	et il monta au ciel; il est assis à la droite du Père.
Il reviendra dans la gloire, pour juger les vivants et les morts
	et son règne n’aura pas de fin.
Je crois en l’Esprit Saint, qui est Seigneur et qui donne la vie;
	il procède du Père et du Fils.
Avec le Père et le Fils, il reçoit même adoration et même gloire;
	il a parlé par les prophètes.
Je crois en l’Église, une, sainte, catholique et apostolique.
Je reconnais un seul baptême pour le pardon des péchés.
J’attends la résurrection des morts, et la vie du monde à venir.}
\end{alltt}

\smalltitle{Offertoire}

\gscore{TO30Of}
\translation{\rubric{D'après le ps. 118.} Seigneur, fais-moi vivre selon ta parole : je connaîtrai tes exigences.\\
\vv \rubrique{\emph{1. }} Agis pour ton serviteur selon ton amour: n'ôte pas de ma bouche la parole de vérité.\\
\vv \rubrique{\emph{2. }} Éclaire-moi, que j'apprenne tes volontés: accepte en offrande ma prière, Seigneur.}

\smalltitle{Prière sur les offrandes}
\begin{paracol}{2}

\twocoltext{\vv Oráte, fratres:
ut meum ac vestrum sacrifícium
acceptábile fiat apud Deum Patrem omnipoténtem.\\
\rr Suscípiat Dóminus sacrifícium de mánibus tuis
ad laudem et glóriam nóminis sui,
ad utilitátem quoque nostram
totiúsque Ecclésiæ suæ sanctæ.}{
\vv Priez, frères et sœurs : que mon sacrifice, et le vôtre, soit agréable à Dieu le Père tout-puissant.\\
\rr Que le Seigneur reçoive de vos mains ce sacrifice à la louange et à la gloire de son nom, pour notre bien et celui de toute l’Église.}

\twocoltext{Réspice, quǽsumus, Dómine,
múnera quæ tuæ offérimus maiestáti,
ut, quod nostro servítio géritur,
ad tuam glóriam pótius dirigátur.
Per Christum Dóminum nostrum.\\
\rr Amen.}{
TODO.\\
\rr Amen.}

\smalltitle{Préface}

\smallscore{MOR10_PrefaceOF}
\emph{\vvrub Le Seigneur soit avec vous. \rrrub Et avec votre esprit.\\
\vvrub Élevons notre cœur. \rrrub Nous le tournons vers le Seigneur.\\
\vvrub Rendons grâce au Seigneur notre Dieu. \rrrub Cela est juste et bon.}

\twocoltext{Vere dignum et iustum est, TODO}{
TODO}

\smalltitle{Sanctus}

\gscore{KY11S}
\emph{Saint, Saint, Saint, le Seigneur, Dieu de l'Univers. Le ciel et la terre sont remplis de ta gloire. Hosanna au plus haut des cieux. Béni soit celui qui vient au nom du Seigneur. Hosanna au plus haut des cieux.}

\rubric{Canon romain et Notre Père: voir page TODO}

\smalltitle{Agnus Dei}

\gscore{KY11A}

\smalltitle{Communion}

\twocoltext{
Ecce Agnus Dei, ecce qui tollit peccáta mundi.
Beáti qui ad cenam Agni vocáti sunt.}{
Voici l’agneau de Dieu,
voici celui qui enlève les péchés du monde.
Heureux les invités au repas des noces de l’Agneau !}

\twocoltext{Dómine, non sum dignus, ut intres sub téctum meum,
sed tantum dic verbo, et sanábitur ánima mea.}{
Seigneur, je ne suis pas digne de te recevoir, mais dis seulement une parole et je serai guéri.}

\gscore{TO30Co}
\translation{\rubrique{Ps. 19: 6, 2, 3, 4, 5, 7} Nous acclamerons ta victoire en arborant le nom de notre Dieu.\\
\vv \rubrique{\emph{2. }} Que le Seigneur te réponde au jour de détresse, que le nom du Dieu de Jacob te défende.\\
\vv \rubrique{\emph{3. }} Du sanctuaire, qu'il t'envoie le secours, qu'il te soutienne des hauteurs de Sion.\\
\vv \rubrique{\emph{4. }} Qu'il se rappelle toutes tes offrandes ; ton holocauste, qu'il le trouve savoureux.\\
\vv \rubrique{\emph{5. }} Qu'il te donne à la mesure de ton coeur, qu'il accomplisse tous tes projets.\\
\vv \rubrique{\emph{7. }} Le Seigneur accomplira toutes tes demandes : maintenant, je le sais : le Seigneur donne la victoire à son messie.}

\smalltitle{Postcommunion}

\twocoltext{
\vv Orémus.\\
Perfíciant in nobis, Dómine, quǽsumus,
tua sacraménta quod cóntinent,
ut, quæ nunc spécie gérimus,
rerum veritáte capiámus.
Per Christum Dóminum nostrum.\\
\rr Amen.}{
\vv Prions le Seigneur.\\
TODO.\\
\rr Amen.}

\smalltitle{Envoi}

\smallscore{MOR25_Benedicat}
