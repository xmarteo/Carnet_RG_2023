% !TEX TS-program = lualatex
% !TEX encoding = UTF-8

\documentclass[RG2023_CarnetCommun.tex]{subfiles}

\ifcsname preamble@file\endcsname
  \setcounter{page}{\getpagerefnumber{M-20231029_4_Laudes}}
\fi

\begin{document}

\bigtitle{30\textsuperscript{e} dimanche \emph{per Annum} --- Laudes --- Liturgia Horarum}

\smallscore{DIA}
\translation{\vvrub Dieu, viens à mon aide.
\rrrub Seigneur, viens vite à mon secours.
Gloire au Père, au Fils, et au Saint-Esprit.
Comme il était au commencement, maintenant et toujours, 
et dans les siècles des siècles. Amen. Alléluia.}

\smalltitle{Hymne}
\gscore{F1LH}
\hymntranslation{
Voici que se défont les ombres de la nuit :
rougeoyante, l’aurore étincelle, splendide ;
de toute notre ardeur, ensemble supplions
le Seigneur tout-puissant.

Que Dieu veuille nous prendre en sa miséricorde :
qu’il chasse tout effroi, nous donne le salut.
Qu’il nous accorde aussi, dans sa bonté de Père,
le royaume des cieux.

Que Dieu nous le concède en sa béatitude,
lui qui est Père et Fils et qui est Saint-Esprit,
celui de qui résonne, en l’univers entier,
la louange de gloire !}

\smalltitle{Psaume 117}
\gscore{H2F1LA1}
\translation{Le Seigneur est pour moi, je ne crains pas ; que pourrait un homme contre moi?}
\psalm{117}{8}

\smalltitle{Cantique Dn. 3, ii}
\gscore{H2F1LA2}
\translation{D’une seule voix, les trois jeunes gens priaient dans la fournaise en chantant : 
Dieu soit béni!}
\psalm{dn3-2}{8}

\smalltitle{Psaume 150}
\gscore{H2F1LA3}
\translation{Louez le Seigneur selon sa grandeur!}
\psalm{150}{8}

\smalltitle{Lecture brève}

\rubric{Ez 36: 25-27}
Je répandrai sur vous une eau
pure, et vous serez purifiés ;
de toutes vos souillures, de
toutes vos idoles, je vous purifierai.
Je vous donnerai un coeur
nouveau, je mettrai en vous un
esprit nouveau. J’ôterai de votre
chair le coeur de pierre, je vous
donnerai un coeur de chair. Je
mettrai en vous mon esprit, je ferai
que vous marchiez selon mes
lois, que vous gardiez mes préceptes
et leur soyez fidèles.

\gscore{H2F1LR}
\translation{\rrrub À toi, Dieu, nous rendons grâce, * et nous invoquons ton Nom. \vvrub Nous proclamons tes merveilles.}

\smalltitle{Benedictus}
\gscore{TO30LABA}
\translation{Maître, dans la Loi, quel est le grand commandement ?
Jésus lui répondit : Tu aimeras le Seigneur ton Dieu de tout ton cœur, alléluia.}
\psalm{ben}{8}

\smalltitle{Intercessions}

Salvatóri nostro grátias agámus, qui in hunc mundum
descéndit ut esset Deus nobíscum. Eum invocémus
clamántes:
Christe, Rex glóriæ, sis lux et gáudium nostrum.
Christe Dómine, lux óriens ex alto, primítiæ resurrectiónis
futúræ,
–– da nos te sequi, ne in umbra mortis sedeámus, sed
ut in lúmine vitæ ambulémus.
Bonitátem tuam in ómnibus creatúris diffúsam nobis
osténde,
–– ut tuam ubíque glóriam contemplémur.
Ne patiáris, Dómine, hódie malo nos vinci,
–– sed fac ut in bono nos vincámus malum.
Qui in Iordáne baptizátus, a Sancto Spíritu es inúnctus,
–– da nos hódie grátia agi Sancti Spíritus tui.


R
1329
endons grâce à notre Sauveur
qui est venu en ce monde
pour être ” Dieu avec nous”
Invoquons-le en chantant :
􀀽 R. Ô Christ, Roi de gloire, sois
notre lumière et notre joie.
Seigneur Jésus-Christ, lumière
d’en haut, prémices de la résurrection
future, — prends-nous
à ta suite pour que nous n’habitions
pas l’ombre de la mort,
mais que nous marchions dans
la lumière de la vie. Ô.
Montre-nous ta bonté, présente
en toute créature, — afin de
contempler partout ta gloire. Ô.
Ne permets pas que nous soyons
vaincus aujourd’hui par le mal :
— fais que nous soyons vainqueurs
du mal par le bien. Ô.
Par ton baptême dans le Jourdain,
tu as été consacré par
l’Esprit Saint : — donne-nous la
grâce d’être conduits par ton Esprit.
Ô.

\rubric{Notre Père, oraison et conclusion, comme aux 2\textsuperscript{e} Vêpres, page \pageref{M-conclusionLV}.}

\end{document}
