% !TEX TS-program = lualatex
% !TEX encoding = UTF-8

\documentclass[RG2023_CarnetCommun.tex]{subfiles}

\ifcsname preamble@file\endcsname
  \setcounter{page}{\getpagerefnumber{M-20231029_1_1Vepres}}
\fi

\begin{document}

\bigtitle{30\textsuperscript{e} dimanche \emph{per Annum} --- 1\textsuperscript{e} Vêpres --- Liturgia Horarum}

\smallscore{DIA}
\translation{\vvrub Dieu, viens à mon aide.
\rrrub Seigneur, viens vite à mon secours.
Gloire au Père, au Fils, et au Saint-Esprit.
Comme il était au commencement, maintenant et toujours, 
et dans les siècles des siècles. Amen. Alléluia.}

\smalltitle{Hymne}
\gscore{F11VH}
\hymntranslation{
Ô Dieu, source de toutes choses,
par l’acte de ta création
tu as comblé tout l’univers
de la richesse de tes dons ;

Au terme d’un si grand ouvrage,
tu as voulu te reposer
afin d’instaurer le repos
parmi nos tâches d’ici-bas.

Accorde maintenant aux hommes
de pleurer sur toutes leurs fautes,
de s’attacher à la vertu
et d’être heureux de tes bienfaits.

Ainsi, lors de l’ultime angoisse,
devant le juge redoutable,
nous goûterons la même joie,
comblés par le don de ta paix.

Exauce-nous, Père très bon,
et toi, le Fils égal au Père,
avec l’Esprit Consolateur
régnant pour les siècles des siècles.}

\smalltitle{Psaume 118, xiv}
\gscore{H2F11VA1}
\translation{Donne la lumière à mes yeux, Seigneur.}
\psalm{118-14}{8}

\smalltitle{Psaume 15}
\gscore{H2F11VA2}
\translation{Garde-moi, Seigneur : j’ai fait de toi mon refuge.}
\psalm{15}{2}

\smalltitle{Cantique Ph. 2}
\gscore{H2F11VA3}
\translation{Le Seigneur lui a donné une gloire sans fin et, pour héritage, un Nom éternel.}
\psalm{ph2}{1}

\smalltitle{Lecture brève}

\rubric{Col 1: 2b-6a}

À vous, la grâce et la paix de
la part de Dieu notre Père.
Nous rendons grâce à Dieu, le
Père de notre Seigneur Jésus
Christ, en priant pour vous à tout
moment. Nous avons entendu
parler de votre foi dans le Christ
Jésus et de l’amour que vous avez
pour tous les fidèles dans l’espérance
de ce qui vous est réservé
au ciel ; vous en avez déjà reçu
l’annonce par la parole de vérité,
l’Évangile qui est parvenu jusqu’à
vous. Lui qui porte du fruit
et progresse dans le monde entier,
il fait de même chez vous.

\gscore{H2F11VR}
\translation{\rrrub Du levant au couchant du soleil, * loué soit le Nom du Seigneur. \vvrub Sa gloire domine les cieux.}

\smalltitle{Magnificat}
\gscore{TO301VAM}
\translation{La Sagesse a bâti sa maison, elle a taillé sept colonnes; 
de la terre entière, de tout peuple et de toute nation elle a fait son domaine.}
\psalm{magn}{7}

\smalltitle{Intercessions}

Deus plebem, quam elégit in hereditátem suam, ádiuvat
et tuétur, ut sit beáta. Ei grátias agámus et pietátis
eius mémores clamémus:
In te, Dómine, confídimus.

Orámus te, Pater clementíssime, pro Papa nostro N. et
Epíscopo nostro N.,
–– prótege illos et tua virtúte sanctífica.
(In te, Dómine, confídimus.)
Infírmi se Christi sócios passiónum séntiant
–– et consolatiónis eius semper partícipes.
Réspice in tua pietáte tecto caréntes,
–– ut locum dignæ habitatiónis váleant inveníre.
Fructus terræ dare et conserváre dignéris,
–– ut panem omnes cotidiánum repériant.
Tu, Dómine, magna defúnctos pietáte proséquere,
–– mansiónem eis concéde cæléstem.

Dieu secourt et protège le peuple
qu’il s’est choisi pour le rendre
bienheureux. Faisons mémoire
de ses bienfaits et rendons-lui
grâce en chantant :
􀀽 R. Seigneur, nous avons
confiance en toi.
Nous te prions, Père infiniment
bon, pour notre pape N. et notre
évêque N. : — protège-les et
sanctifie-les par ta puissance. Seigneur.
Que les malades se sentent unis
au Christ dans sa passion, — et
qu’ils reçoivent toujours son réconfort.
Seigneur.
Dans ta bonté, protège les sans-abris
: — qu’ils trouvent une
demeure pour vivre dignement.
Seigneur.
Accorde en abondance les fruits
de la terre, — pour que nul ne
manque du pain quotidien. Seigneur.
Dans ta grande miséricorde, Seigneur,
— accueille les défunts
dans la demeure céleste. Seigneur.

\rubric{Notre Père, oraison et conclusion, comme aux 2\textsuperscript{e} Vêpres, page \pageref{M-conclusionLV}.}

\end{document}